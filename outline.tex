% !TEX program = pdflatex
\RequirePackage[l2tabu, orthodox]{nag}
\documentclass{article}

% FONTS
\usepackage[T1]{fontenc}

% Replace default Latin Modern typewriter with its proportional counterpart
% http://www.tug.dk/FontCatalogue/lmoderntypewriterprop/
\renewcommand*\ttdefault{lmvtt}


%%% OPTION 1 - Fourier Math + New Century Schoolbook + ParaType Sans

% % Import Fourier Math (this imposes its own New Century Schoolbook type)
% % http://www.ctan.org/tex-archive/fonts/fouriernc/
% \usepackage{fouriernc}
% \usepackage{amsmath}
% % Replace with TeX Gyre Schola version of New Century Schoolbook (must scale!)
% % http://www.tug.dk/FontCatalogue/tgschola/
% \usepackage[scale=0.92]{tgschola}
% \usepackage[scaled=0.88]{PTSans}

%% OPTION 2 - MathDesign Math + Bitstream Charter + ParaType Sans

% Import MathDesign (this brings along Bitstream Charter)
% http://www.ctan.org/tex-archive/fonts/mathdesign/
\usepackage[bitstream-charter]{mathdesign}
\usepackage{amsmath}
\usepackage[scaled=0.92]{PTSans}


% %%% OPTION 3 - MTPRO 2 Math + Termes Times + ParaType Sans

% \usepackage{tgtermes}
% \usepackage{amsmath}
% \usepackage[subscriptcorrection,
%             amssymbols,
%             mtpbb,
%             mtpcal,
%             nofontinfo  % suppresses all warnings
%            ]{mtpro2}
% \usepackage{scalefnt,letltxmacro}
% \LetLtxMacro{\oldtextsc}{\textsc}
% \renewcommand{\textsc}[1]{\oldtextsc{\scalefont{1.10}#1}}
% \usepackage[scaled=0.92]{PTSans}

% GEOMETRY
\usepackage[
  paper  = letterpaper,
  left   = 1.65in,
  right  = 1.65in,
  top    = 1.0in,
  bottom = 1.0in,
  ]{geometry}

% COLOR
\usepackage[usenames,dvipsnames]{xcolor}
\definecolor{shadecolor}{gray}{0.9}

% SPACING and TEXT
\usepackage[final,expansion=alltext]{microtype}
\usepackage[english]{babel}
\usepackage[parfill]{parskip}
\usepackage{afterpage}
\usepackage{framed}

%redefine the leftbar environment to accept a width and coloring options
\renewenvironment{leftbar}[1][\hsize]
{%
  \def\FrameCommand
  {%
    {\color{Gray}\vrule width 3pt}%
    \hspace{10pt}%
    %\hspace{0pt}\fboxsep=\FrameSep\colorbox{black!10}%
  }%
  \MakeFramed{\hsize#1\advance\hsize-\width\FrameRestore}%
}%
{\endMakeFramed}

% define a paragraph header function
\DeclareRobustCommand{\parhead}[1]{\textbf{#1}~}

% EDITING
% line numbering in left margin
\usepackage{lineno}
\renewcommand\linenumberfont{\normalfont
                             \footnotesize
                             \sffamily
                             \color{SkyBlue}}
% ragged paragraphs in right margin
\usepackage{ragged2e}
\DeclareRobustCommand{\sidenote}[1]{\marginpar{
                                    \RaggedRight
                                    \textcolor{Plum}{\textsf{#1}}}}
% paragraph counter in right margin
\newcommand{\parnum}{\bfseries\P\arabic{parcount}}
\newcounter{parcount}
\newcommand\p{%
    \stepcounter{parcount}%
    \leavevmode\marginpar[\hfill\parnum]{\parnum}%
}
% paragraph helper
\DeclareRobustCommand{\PP}{\textcolor{Plum}{\P} }

% COUNTERS
\renewcommand{\labelenumi}{\color{black!67}{\arabic{enumi}.}}
\renewcommand{\labelenumii}{{\color{black!67}(\alph{enumii})}}
\renewcommand{\labelitemi}{{\color{black!67}\textbullet}}

% FIGURES
\usepackage{graphicx}
\usepackage[labelfont=bf]{caption}
\usepackage[format=hang]{subcaption}

% TABLES
\usepackage{booktabs}

% ALGORITHMS
\usepackage[algoruled]{algorithm2e}
\usepackage{listings}
\usepackage{fancyvrb}
\fvset{fontsize=\normalsize}

% BIBLIOGRAPHY
\usepackage{natbib}

% HYPERREF
\usepackage[colorlinks,linktoc=all]{hyperref}
\usepackage[all]{hypcap}
\hypersetup{citecolor=BurntOrange}
\hypersetup{linkcolor=MidnightBlue}
\hypersetup{urlcolor=MidnightBlue}

% CLEVEREF must come after HYPERREF
\usepackage[nameinlink]{cleveref}

% ACRONYMS
\usepackage[acronym,smallcaps,nowarn]{glossaries}
% \makeglossaries

% COLOR DEFINITIONS
\newcommand{\red}[1]{\textcolor{BrickRed}{#1}}
\newcommand{\orange}[1]{\textcolor{BurntOrange}{#1}}
\newcommand{\green}[1]{\textcolor{OliveGreen}{#1}}
\newcommand{\blue}[1]{\textcolor{MidnightBlue}{#1}}
\newcommand{\gray}[1]{\textcolor{black!60}{#1}}

% LISTINGS DEFINTIONS
\lstdefinestyle{mystyle}{
    commentstyle=\color{OliveGreen},
    keywordstyle=\color{BurntOrange},
    numberstyle=\tiny\color{black!60},
    stringstyle=\color{MidnightBlue},
    basicstyle=\ttfamily,
    breakatwhitespace=false,
    breaklines=true,
    captionpos=b,
    keepspaces=true,
    numbers=left,
    numbersep=5pt,
    showspaces=false,
    showstringspaces=false,
    showtabs=false,
    tabsize=2
}
\lstset{style=mystyle}


\DeclareRobustCommand{\mb}[1]{\ensuremath{\boldsymbol{\mathbf{#1}}}}

\DeclareMathOperator*{\argmax}{arg\,max}
\DeclareMathOperator*{\argmin}{arg\,min}

\DeclareRobustCommand{\KL}[2]{\ensuremath{\textrm{KL}\left(#1\;\|\;#2\right)}}

\newcommand{\mbx}{\mathbold{x}}
\newcommand{\mbX}{\mbf{X}}

\newcommand{\mbz}{\mathbold{z}}

\newcommand{\mbI}{\mbf{I}}

\newcommand{\mbZ}{\mbf{Z}}
\newcommand{\mbL}{\mbf{L}}

\newcommand{\mbtheta}{\mathbold{\theta}}
\newcommand{\mbTheta}{\mathbold{\Theta}}
\newcommand{\mbomega}{\mathbold{\omega}}
\newcommand{\mbOmega}{\mathbold{\Omega}}
\newcommand{\mbsigma}{\mathbold{\sigma}}
\newcommand{\mbSigma}{\mathbold{\Sigma}}

\newcommand{\mblambda}{\mathbold{\lambda}}
\newcommand{\mbgamma}{\mathbold{\gamma}}
\newcommand{\mbzeta}{\mathbold{\zeta}}
\newcommand{\mbeta}{\mathbold{\eta}}
\newcommand{\mbbeta}{\mathbold{\beta}}
\newcommand{\mbphi}{\mathbold{\phi}}
\newcommand{\mbmu}{\mathbold{\mu}}
\newcommand{\mbrho}{\mathbold{\rho}}

\newcommand\dif{\mathop{}\!\mathrm{d}}
\newcommand{\diag}{\textrm{diag}}
\newcommand{\supp}{\textrm{supp}}

\newcommand{\E}{\mathbb{E}}
\newcommand{\Var}{\mathbb{V}\textrm{ar}}

\newcommand{\bbN}{\mathbb{N}}
\newcommand{\bbZ}{\mathbb{Z}}
\newcommand{\bbR}{\mathbb{R}}
\newcommand{\bbS}{\mathbb{S}}

\newcommand{\cL}{\mathcal{L}}

\newcommand{\cN}{\mathcal{N}}
\newcommand{\Gam}{\textrm{Gam}}
\newcommand{\InvGam}{\textrm{InvGam}}

\newacronym{KL}{kl}{Kullback-Leibler}
\newacronym{ELBO}{elbo}{\emph{evidence lower bound}}
\newacronym{POPELBO}{pop-elbo}{\emph{population evidence lower bound}}

\newacronym{SVI}{svi}{stochastic variational inference}
\newacronym{BUMPVI}{bump-vi}{bumping variational inference}

\newacronym{GMM}{gmm}{Gaussian mixture model}
\newacronym{LDA}{lda}{latent Dirichlet allocation}

\newacronym{SUTVA}{sutva}{stable unit treatment value assumption}


% \linenumbers

\title{Trygve's Problem}
\author{Jackson Loper}

\usepackage{amsthm}
\newtheorem{thm}{Theorem}

\begin{document}
\maketitle

\section{Overview}

When many different measurement tools are available, it can be difficult to understand
how to combine information from different tools.  This challenge is compounded when the measurement
tools are destructive.  For example, after processing a neuron with a given single-cell rna
sequencing technique, we cannot generally go back and process that same neuron with a different
technique.  This is because the measurement technique destroys the original neuron.  This means
we can never see \emph{both measurements} for the same cell, side by side.  How, then, can we ever
hope to understand how the different measurements relate to each other?

In this paper, we hope to shed some light on this this basic mathematical question.  It has many
applications in the hard sciences.  For example:

\begin{itemize}

\item Different tools may teach us different things, and we need to combine tools
to learn everything we can.  Let $A,B$ denote two different quantities of interest about cells (e.g. morphology and gene expression).   Let us say one tool enables us to measure $A$ and another tool to measure $B$.  We would like
to be able to say something about what morphologies are associated with what gene expressions.  

\item Two labs may used different methods, but may wish to pool results.  Each lab looks at the other
lab's data and asks themselves "what would their data have looked like if they had used our method?"  When
they find they don't know the answer, it greatly complicates collaboration.

\item One may attempt to cluster the data.  It is simple to cluster the data from one technique, and cluster the data
from another technique -- but how can we know how those clusters correspond to one another?  

\end{itemize}

How could we learn the relationship between two measurement techniques, unless we can observe both measurements 
applied to the same members of a population (dual measurement)?  In this work we show that it is indeed 
sometimes possible to get useful and statistically rigorous bounds, even without these dual measurements.   

In particular, we consider the case that we can obtain samples from a variety of subpopulations.  
If we can assume that the \emph{relationship} between the measurements is the same in each subpopulation, 
we can use these samples to learn something about that relationship -- even though we never see both 
measurements for any single member of the population.  

We then apply our method to single cell RNA data.  We show how different clusterings from different single 
cell RNA modalities can be connected.  Our method yields results which are generally consistent with the scientist's
understanding of the data and clusters, but also reveal potential gaps which could be important for further study.

\section{Mathematical formulation}

Mathematically, we formulate our model as follows:

\begin{itemize}
\item Let us say we have $n+m$ individual members of a population.  These could be cells, humans, or whatever
the smallest sample unit may be for a given problem.
\item For each individual member, we have some readily observable features by which we can group the members,
such as the size of a cell, where an individual lives, or some other basic information.  We designate these
features as $\ell_i$ for the $i$th individual.
\item For the first $n$ members, we have made an observation using the first technique.  This gives us
the values $X_1 \cdots X_n$.
\item The last $m$ members were observed using the second technique.  This gives us the values
$Y_{n+1} \cdots Y_{n+m}$.
\end{itemize}

For simplicity, we here assume that $\ell_i,X_i,Y_i$ take values in finite sets, $\Omega_\ell,\Omega_X,\Omega_Y$.  Extensions to more general
cases should be straightforward, but we must leave them to future work.  

We further posit a set of \emph{counterfactual} variables -- variables which could not ever be obtained in practice,
but which help us organize our thinking.  These are sometimes referred to as "potential outcomes"  (cf. \cite{rubin2005causal}).

\begin{itemize}
\item Let $Y_1 \cdots Y_n$ denote the observations we \emph{would have gotten} if we had applied the second technique
to the first $n$ members.
\item Let $X_{n+1} \cdots X_n$ denote the observations we would have gotten if we had applied the first technique
to the last $m$ members. 
\end{itemize}

Our main assumption is that 

\begin{equation}
\mathbb{P}(X_i=x,Y_i=y|L_i=\ell)=p^*(x_i|\ell_i)q^*(y_i|x_i)
\end{equation}

for some distributions $p^*,q^*$.  Moreover, we assume that the vectors $\{(p^*(x_1|\ell)\cdots p^*(x_n|\ell))\}_\ell$ 
are linearly independent.  If this seems unlikely, it may make sense to collapse similar values of $\ell$ together.

The validity of these assumptions for a given situation should be closely contemplated.  
There are several key questions to answer
when deciding whether this assumption is applicable.  Does the process by which samples are gathered
gathered depend only upon $\ell$?  In particular, is it not at all statistically related to which measurement technique was 
applied, except through $\ell$?  Are the particular measurement biases of both techniques the same for every value of $\ell$?  
Is the statistical relationship between the quantities being measured by the 
two techniques the same for every value of $\ell$?\footnote{Note that this is 
automatically true if both techniques are measuring the same thing.  More generally, if two techniques have something that
they both measure, we can restrict attention to that one common phenomenon.}  If the answer to all of these questions is yes, the assumption that $(X_i,Y_i)$ is drawn from $p(x_i|\ell_i)p(y_i|x_i)$ may apply.

We can now articulate our goal.  Informally, our goal is to see what we can understand about the relationship between
the observations $X_i$ from one measurement modality and the other measurement modality $Y_i$.  Ideally, we would
like to learn the distribution $p^*(y_i|x_i)$.  Unfortunately, given the limitations of the data, this may not be 
possible.  Instead, we articulate a slightly less amibtious goal: to put \emph{bounds} on $q(y_i|x_i)$.  Specifically,
we will seek to produce functions $\hat u_1(y|x),\hat u_2(y|x)$ such that

\begin{equation}\label{eq:ubound}
\hat u_1(y|x) \leq q^*(y|x) \leq \hat u_2(y|x)
\end{equation}

In particular, we define

\begin{align*}
\Theta^* &\triangleq\left\{q:\ \sum_x p^*(x|\ell) q^*(y|x) =\sum_x p^*(x|\ell)q^*(y|x), \forall \ell,y\right\}\\ 
u_1^*(y|x) &= \inf_{q\in\Theta^*} \{q(y|x)\} \\
u_2^*(y|x) &= \sup_{q\in\Theta^*} \{q(y|x)\}
\end{align*}

Then obviously $u_1^*(y|x) \leq q^*(y|x)\leq u_2^*(y|x)$ is a valid bound.  Moreover, given the data and assumptions
we have, it is as tight a bound as is possible to make (indeed, we could not hope to
distinguish between any two $q_1,q_2 \in \Theta^*$, since they yield the same distributions on what we can
observe).

Thus, our mathematical task to to attempt to find estimators for $u_1^*,u_2^*$, which will allow us to bound
the values of $q^*(y|x)$, and thus understand something about the relationship between the two measurements, $X$ and
$Y$.

\section{Mathematical solution}

To produce useful bounds on $q^*(y|x)$, we first estimate

\begin{itemize}
\item $p^*$ using the empirical distribtion of the observations $\{X_i\}$, $\hat p(x|\ell)$
\item $h^*(y|\ell)\triangleq \sum_x p^*(x|\ell)q^*(y|\ell)$ using the empirical distribtion of $\{Y_i\}$, $\hat h(y|\ell)$
\end{itemize}

These empirical distributions may not be quite consistent with the original assumption that the distribution
of $X,Y|\ell$ may be written as $p^*(x|\ell)q^*(y|x)$.  In particular, it may be that there is \emph{no} value of $q$ such
that $\sum_x \hat p(x|\ell) q(y|x) = \hat h(y|\ell)$.  There may also be many such values.  We satisfy ourselves
with finding a somewhat good fit for $q$.  Let 
$N_{X,\ell}=\#\{i\leq n:\ \ell_i=\ell\},N_{Y,\ell}=\#\{i> n:\ \ell_i=\ell\}$.  We take

\begin{equation}
\hat q = \argmax_q \sum_\ell N_{Y,\ell}\sum_{y}\hat h(y|\ell)\log\left(\sum_{x}\hat p(x|\ell)q(y|x)\right) + \kappa \sum_{xy} \log q(y|x)
\end{equation}

Note in the second term we apply an entropic penalty to $q$.  This is both necessary to ensure 
that $\hat q$ is uniquely defined in terms of $\hat q,\hat h$, as well as a useful regularization.
This regularization pushes us slightly towards uniform distributions, which makes sense if we believe that $q(y|x)>0$ 
for each $x,y$.  Other regularizations could also be used and would potentially
enjoy similar theoretical guarantees to the ones we will show below.

This method gives us an estimate for $\hat q$, but we emphasize this estimate may be \emph{inconsistent}
for the true value of $q^*$.  As we have said, this is simply a limitation of the data we have available -- attempting to deduce
$q^*$ may be too much to ask for.  However, we can get bounds.  

The key is to search over all possible values of $q$ which agree with $\hat q$ in terms of what \emph{is} identifiable.
That is, let 

\begin{align*}
\hat\Theta &\triangleq\left\{q:\ \sum_x \hat p(x|\ell)\hat q(y|x) =\sum_x \hat p(x|\ell)q(y|x), \forall \ell,y\right\}\\ 
\hat u_1(y|x) &= \inf_{q\in\hat\Theta} \left\{q(y|x) \right\} \\
\hat u_2(y|x) &= \sup_{q\in\hat\Theta} \left\{q(y|x) \right\}
\end{align*}

These will serve as consistent estimators for our quantities of interest, $u_1^*,u_2^*$, as the following theorem shows.\vspace{.1in}

\begin{thm}
Fix any $\kappa>0$.  Let $N_{X,\ell},N_{Y,\ell}\rightarrow\infty$ uniformly in $\ell$.  Then $\hat u_1(y|x)\rightarrow u^*_1(y|x)$ in probability, and likewise for $u_2$.
\end{thm}

\begin{proof}

The logic is in perfect symmetry for $u_1,u_2$, so we focus on the $u_1$ case.

We begin by seeing how our process would have worked if we were able to know the exact values of $p^*$ and $h^*$,
instead of the empirical distributions $\hat p,\hat h$.  That is, we define
%
\begin{align*}
L_N(q,p,h) &\triangleq \sum_\ell N_{Y,\ell}\sum_{y}h(y|\ell)\log\left(\sum_{x}p(x|\ell)q(y|x)\right) + \kappa \sum_{xy} \log q(y|x) \\
\Theta(p,\tilde q) &\triangleq\left\{q:\ \sum_x p(x|\ell) \tilde q(y|x) =\sum_x p(x|\ell) q(y|x), \forall \ell,y\right\}
\end{align*}
%
and take
%
\begin{align*}
\tilde q^* &= \argmax_q L_N(q,p^*,h^*)  \\
\tilde u_1^*(y|x) &= \inf_{q\in \Theta(p^*,\tilde q^*)} q(y|x) \\
\end{align*}
%
Our first task is to show that the map $p^*,h^* \mapsto \tilde u_1^*$ is continuous.  This follows in two steps:

%%%%% two steps
\begin{enumerate}
% STEP I
\item The map $p^*,h^* \mapsto \argmax_q L(q,p^*,h^*)$ is continuous.  Note that $q\mapsto L(q,p^*,h^*)$
is \emph{uniformly} strictly concave for all $p^*,h^*$.  Indeed, the hessian is given by
%
\begin{align*}
\frac{\partial L_N(q,p^*,h^*)}{\partial q(x_1,y_1) \partial q(x_2,y_2)} =& 
    - \delta_{y_1=y_2}\sum_\ell N_{Y,\ell} h^*(y_1|\ell) 
              \frac{p^*(x_1|\ell)p^*(x_2|\ell)}{\left(\sum_x p(x|\ell) q(y_1|x)\right)^2} \\
    & \qquad - 2\kappa \delta_{(x_1,y_1)=(x_2,y_2)} \\
\end{align*}
%
Since $q$ lives in a compact space and $L$ is continuous with respect to $p^*,h^*$, we thus 
know that $p^*,h^* \mapsto \argmax_q L_N(q,p^*,h^*)$ is also continuous.  
% STEP II
\item The map $p,\tilde q \mapsto \inf_{q\in\Theta(p,\tilde q)} q(y|x)$ is continuous, as long as 
$\{(p(x_1|\ell)\cdots p(x_n|\ell))\}_\ell$ are linearly independent (remember we have assumed the independence). 
It would suffice to show the following: 
for any fixed $p_1 \approx p_2,\tilde q_1 \approx \tilde q_2,x,y$ where the $\approx$ is sufficiently tight
and $q_1\in \Theta(p_1,\tilde q_1)$, 
we can find $q_2\in \Theta(p_2,\tilde q_2)$ with $q_1(y|x)\approx q_2(y|x)$.  We will solve
this problem first by looking at what happens if $\tilde q$ changes, and then at what happens if $p$ changes.
\begin{itemize}
\item Let us say $p_1=p_2=p$.  Then, in the absence of constraints, we could solve
our problem by simply taking $q_2 = q_1 + \tilde q_2 - \tilde q_1$.
However, if $q_1(y|x)=0$ for any $x,y$, we cannot guarantee that the resulting $q_2$ will be a positive measure. 
To resolve this, we note that the entropic penalty ensures that $\tilde q_2,\tilde q_1$ lie on the interior
of the probability simplex.  This enables us to resolve the issue by constructing ``smoothed'' version of $q_1$, namely
$(1-\alpha) q_1 + \alpha (\tilde q_1)$ which also lies on the interior of the probability simplex.   Let's do this
rigorously.

Fix any $\tilde q_1,\delta>0$ and $q_1 \in \Theta(p,\tilde q_1)$. Let $c = \min_{x,y} \tilde q_1(y|x)$.  Now pick any $\tilde q_2$ such that $|\tilde q_2-\tilde q_1|< \delta c/2$ in the uniform norm.  Take $q_2 = (1-\delta)q_1 + \delta \tilde q_1 + \tilde q_2 - \tilde q_1$.  Observe that $q_2 (y|x) \geq \delta c - \delta c/2$ for every $x,y$, i.e. it is a valid
probability density.  Note also that $q_2 \in \Theta(p, \tilde q_2)$.  Finally, note that $|q_2 - q_1|<\delta$ in
the uniform norm, as desired. 

\item Let us now say $\tilde q_1 = \tilde q_2 = \tilde q$.  We are given $q_1,p_1,p_2$ and asked to
produce $q_2\in \Theta(p_2,\tilde q)$ such that $q_1\approx q_2$.  Let 
$A_{1,\ell,x} = p_1^*(x|\ell)$ and $A_{2,\ell,x} = p_2^*(x|\ell)$.  We have assumed that the rows of $A_1$ are linearly independent, so $A_1 A_1^T$ is invertible.  There is therefore an open set around $A_1$ so that as long as $A_2 \approx A_1$ we have that the rows of $A_2$ are also linearly independent.  Therefore to find $q_2 \in \Theta(p_2, \tilde q)$
we can simply take $q_2 = q_1 - A_2^T (A_2 A_2^T)^{-1} A_2 (q_1 -\tilde q)$.  Note that $A_1 (q_1 - \tilde q) =0$, so
the norm of $A_2 (q_1 -\tilde q)$ can be bounded.  Likewise the strength of $(A_2 A_2^T)^{-1}$
can be bounded by the smallest absolute eigenvalue of $A_2$, which, in turn, is bounded by its proximity to $A_1$.  
Finally, the entries of $A_2^T$ are uniformly bounded between zero and one.  In conclusion, 
for any $p_1$ we can always find a $p_2$ close enough so that we can completely control $|q_1-q_2|$ in the Euclidean
norm.  Since we are working in finite dimensions, this is sufficient to control it in the uniform norm.  

Finally, we note that it is possible this method will result in $q_2$ being an invalid probability distribution, i.e.
$q_2(y|x)<0$.  This could be true in two senses.  First, it might be that $\sum_y q_2(y|x)\neq 1$.  Fortunately,
this is impossible.  Indeed, $(q_1 - \tilde q)1 =0$ so certainly $(q_2-q_1)1=(\cdots)(q_1-\tilde q)1=0$ as well.   
Second, we might have that $q_2(y|x)<0$.  To remedy this
second point, we take recourse to the same method explained in detail for the case $p_1=p_2$ above. 
That is, we work with a slightly ``smoothed'' version of $q_1$, and find something which is very close to the smoothed version,
and thereby ensure that the resulting $q_2$ is valid.  This $q_2$ is close to the slightly smoothed $q_1$ and thus also
close to $q_1$.

\end{itemize}

\end{enumerate}

It is well-known that the empirical distributions $\hat p, \hat h$ converge uniformly in probability to the true values, $p^*,h^*$.  Since the map $p,q \rightarrow u_1$ is continuous in a neighborhood of $p^*,q^*$, it follows that $\hat u_1$ also converge uniformly to $\tilde u_1^*$ in probability.

It thus suffices to show that $\tilde u_1^* \rightarrow u_1^*$.  To show this, let $\tilde h^* =p^*\tilde q^*$,
i.e. $h^*(y|\ell)=\sum_x p^*(x|\ell)\tilde q^*(y|x)$.  The key will be to control $\tilde h^*$ and see how that control
enables us to control $\tilde u^*$.

\begin{enumerate}
\item $\tilde h^* \rightarrow h^*$ in the uniform norm as 
$N_{Y,\ell}\rightarrow\infty$ uniformly in $\ell$.

To show this, we express $L$ in terms of 
Kullback-Leibler divergences, $D(h^*_\ell||\tilde h^*_\ell)$ and entropies 
$\mathcal H(h^*_\ell)$.  Applying Pinsker's inequality, we obtain
%
\begin{align*}
L(\tilde q^*,p^*,h^*) &= \sum_\ell N_{y,\ell} (\mathcal H(h^*_\ell) - D(h^*_\ell||\tilde h^*_\ell)) + \kappa \sum_{xy} \log q^*(y|x)\\
&\leq \sum_\ell N_{y,\ell} (\mathcal H(h^*_\ell) - 2|h^*_\ell-\tilde h^*_\ell|^2_{TV} + \kappa \sum_{xy} \log q^*(y|x)\\
&\leq \sum_\ell N_{y,\ell} (\mathcal H(h^*_\ell) - \frac{1}{2}(\sup_y |h^*(y|\ell)-\tilde h^*(y|\ell)|)^2 + \kappa \sum_{xy} \log q^*(y|x)\\
\end{align*}

Since $\tilde q^*,\tilde h^*$ are found by taking $L$ as large as possible, it would thus seem that we could use
this inequality to argue that $h^* \approx \tilde h^*$ when $N_{Y,\ell}$ are large.  There is, however, once again 
a bit of trickiness in the case that $q^*(y|x)=0$ for some $x,y$.  Indeed, it may be that $L(q^*,p^*,h^*)=-\infty$,
and this complicates some of the arguments that we want to make.   To remedy this, we first design a smoothed
version of $q^*$.  Then everything goes through as we might like.  Let's do it rigorously.

Fix $0<\epsilon<1$. We take
$q^*_{SM}= (1-\epsilon) q^* + \epsilon \tilde q^*$ as our smoothed version of $q^*$.  
The convexity of the KL divergence then gives that
%
\begin{align*}
D(h^* || p^*q^*_{SM})&=D(h^*_{\ell} || (1-\epsilon) h^*_{\ell} + \epsilon \tilde h^*) \leq (1-\epsilon)0 + \epsilon D(h^*_{\ell} || \tilde h^*) 
\end{align*}
%
Thus, applying Pinsker (and the fact that $\tilde q^* = \argmax _q L_N(\tilde q,p^*,h^*)$)
%
\begin{align*}
0 &\leq L_N(\tilde q^*,p^*,h^*) - L_N(q^*_{SM},p^*,h^*)\\
 &= \sum_\ell N_{y,\ell} (\epsilon-1) D(h^*_\ell||\tilde h^*_\ell)+ \kappa \sum_{xy} \log \frac{\tilde q^*(y|x)}{q^*_{SM}(y|x)}\\
&\leq \sum_\ell N_{y,\ell} (\epsilon-1) D(h^*_\ell||\tilde h^*_\ell)- \kappa |\Omega_X||\Omega_Y|\log \epsilon \\
&\leq \sum_\ell N_{y,\ell} (\epsilon-1) 2|h^*_\ell-\tilde h^*_\ell|^2_{TV}- \kappa |\Omega_X||\Omega_Y|\log \epsilon \\
&\leq \frac{\epsilon-1}{2}\sum_\ell N_{y,\ell} \left(\sup_y|h^*(y|\ell)-\tilde h^*(y|\ell)|\right)^2-\kappa |\Omega_X||\Omega_Y|\log \epsilon \\
\end{align*}
%
One consequence of that inequality is that 
%
\begin{align*}
\sup_y|h^*(y|\ell)-\tilde h^*(y|\ell)| &\leq \sqrt{\frac{-2\kappa |\Omega_X||\Omega_Y|\log \epsilon}{N_{Y,\ell}} } \\
\end{align*}
%
With this inequality in hand, we are now prepard to prove that $\tilde h^* \rightarrow h^*$.  Pick any $0<\delta<1$.  
Choose $N_{\ell,y}>-2\kappa |\Omega_X||\Omega_Y|\log \delta/\delta$.  Take $\epsilon=\delta$.  Then
%
\begin{align*}
\sup_y|h^*(y|\ell)-\tilde h^*(y|\ell)| &\leq \sqrt{\delta} \\
\end{align*}

\item We now show that if $\tilde h^*\approx h^*$, then $\tilde u_1^* \approx u_1^*$.  To do so, let
%
\[
\tilde \Theta^* = \left\{q: \sum_x p^*(x|\ell)q(y|x)=\sum_x p^*(x|\ell)\tilde q^*(y|x) = \tilde h^*(y|\ell)\right\} 
\]
%
To complete our proof, it then suffices to show that we can find $\tilde h^*$ close enough to $h^*$ so that
we can find a $q \in \tilde \Theta^*$ with $q \approx q^*$.

\end{enumerate}

\end{proof}

Then certainly $u_1^*(y|x) \leq q(y|x)\leq u_2^*(y|x)$.  What's more, we can 

\bibliographystyle{unsrt}
\bibliography{refs}

\end{document}






































